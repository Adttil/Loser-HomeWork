\documentclass[11pt,fancyhdr]{ctexart}
\usepackage{graphicx} 
\usepackage{float}
\usepackage{geometry}
\usepackage{xcolor}
\usepackage{fancyhdr}
\usepackage{minted}
\usepackage{tcolorbox}
\usepackage{pifont}
\usepackage[colorlinks,linkcolor=blue]{hyperref}
\renewcommand{\headrulewidth}{0.2pt}
\renewcommand{\headwidth}{\textwidth}
\renewcommand{\footrulewidth}{0pt}
\geometry{left=2cm,right=2cm,top=3cm,bottom=2cm}
\pagestyle{fancy}
\lhead{\author}
\chead{\date}
\rhead{op}
\lfoot{}
\cfoot{\thepage}
\rfoot{}
\title{现代C++题目}

\author{\href{https://github.com/Mq-b/Loser-HomeWork}{卢瑟帝国}\\}

\date{\today}

\begin{document}
\maketitle

暂时只有 13 道题目,并无特别难度,有疑问可看\href{https://www.bilibili.com/video/BV1Zj411r7eP/}{视频教程}。

% 后面写只需要按照下面这两行的形式就行了,一个 section 标题,一个 input 题目

\section{实现管道运算符}

日期:2023/7/21 出题人:mq白

\begin{minted}[mathescape,	
    linenos,
    numbersep=5pt,
    gobble=2,
    frame=lines,
    framesep=2mm]{c++}
    int main(){
        std::vector v{1, 2, 3};
        std::function f {[](const int& i) {std::cout << i << ' '; } };
        auto f2 = [](int& i) {i *= i; };
        v | f2 | f;
    }
\end{minted}

\begin{tcolorbox}[title = {要求运行结果},
    fonttitle = \bfseries, fontupper = \sffamily, fontlower = \itshape]
    1 4 9
\end{tcolorbox}

$\bullet ~ $难度:\ding{77} \ding{77} \ding{73} \ding{73} \ding{73}

\section{实现自定义字面量 \_f}

日期:2023/7/22 出题人:mq白\\

给出以下代码,在不修改已给出代码的前提下使它满足\textbf{运行结果}。
6 为输入,决定 \textbf{$\pi$} 的小数点后的位数,可自行输入更大或更小数字。


\begin{minted}[mathescape,	
    linenos,
    numbersep=5pt,
    gobble=2,
    frame=lines,
    framesep=2mm]{c++}
    int main(){
        std::cout << "乐 :{} *\n"_f(5);
        std::cout << "乐 :{0} {0} *\n"_f(5);
        std::cout << "乐 :{:b} *\n"_f(0b01010101);
        std::cout << "{:*<10}"_f("卢瑟");
        std::cout << '\n';
        int n{};
        std::cin >> n;
        std::cout << "π:{:.{}f}\n"_f(std::numbers::pi_v<double>, n);
    }
\end{minted}

\begin{tcolorbox}[title = {要求运行结果},
        fonttitle = \bfseries, fontupper = \sffamily, fontlower = \itshape]
    乐 :5 *\\
    乐 :5 5 *\\
    乐 :1010101 *\\
    卢瑟******\\
    6\\
    π:3.141593
\end{tcolorbox}

$\bullet ~ $\textbf{难度}:\ding{72} \ding{72} \ding{73} \ding{73} \ding{73}

$~~~$\textbf{提示}:C++11 用户定义字面量、C++20 format 库。

\end{document}