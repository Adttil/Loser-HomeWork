\documentclass[11pt,fancyhdr]{ctexart}
\usepackage{graphicx} 
\usepackage{float}
\usepackage{geometry}
\usepackage{xcolor}
\usepackage{fancyhdr}
\usepackage{minted}
\usepackage{tcolorbox}
\usepackage{pifont}
\usepackage[colorlinks,linkcolor=blue,bookmarksnumbered=true]{hyperref}
\renewcommand{\headrulewidth}{0.2pt}
\renewcommand{\headwidth}{\textwidth}
\renewcommand{\footrulewidth}{0pt}
\geometry{left=2cm,right=2cm,top=3cm,bottom=2cm}
\pagestyle{fancy}
\lhead{\author}
\chead{\date}
\rhead{op}
\lfoot{}
\cfoot{\thepage}
\rfoot{}
\title{现代C++题目(答案与解析)}


\author{\href{https://github.com/Mq-b/Loser-HomeWork}{卢瑟帝国}\\}

\date{\today}


\newcommand{\hardscore}[1]{
    \newcount\starcount
    \starcount=0
    \loop
        \ifnum\starcount<5
            \ifnum\starcount<#1
            \ding{72}
            \else
            \ding{73}
            \fi
        \advance\starcount by 1
    \repeat
}

\begin{document}
\maketitle

\tableofcontents
\newpage

暂时只有 13 道题目,并无特别难度,有疑问可看\href{https://www.bilibili.com/video/BV1Zj411r7eP/}{视频教程}或答案解析。

% 后面写只需要按照下面这两行的形式就行了,一个 section 标题,一个 input 题目

\section{实现管道运算符}
日期:2023/7/21 出题人:mq白

\begin{minted}[mathescape,	
    linenos,
    numbersep=5pt,
    gobble=2,
    frame=lines,
    framesep=2mm]{c++}
    int main(){
        std::vector v{1, 2, 3};
        std::function f {[](const int& i) {std::cout << i << ' '; } };
        auto f2 = [](int& i) {i *= i; };
        v | f2 | f;
    }
\end{minted}

\begin{tcolorbox}[title = {要求运行结果},
    fonttitle = \bfseries, fontupper = \sffamily, fontlower = \itshape]
    1 4 9
\end{tcolorbox}

$\bullet ~ $难度:\ding{77} \ding{77} \ding{73} \ding{73} \ding{73}

\subsection{答案}

\begin{minted}[mathescape,	
    linenos,
    numbersep=5pt,
    gobble=2,
    frame=lines,
    framesep=2mm]{c++}
    template<typename U, typename F>
        requires std::regular_invocable<F, U&>//可加可不加,不会就不加
    std::vector<U>& operator|(std::vector<U>& v1, F f) {
        for (auto& i : v1) {
            f(i);
        }
        return v1;
    }
\end{minted}

\textbf{不使用模板:}

\begin{minted}[mathescape,	
    linenos,
    numbersep=5pt,
    gobble=2,
    frame=lines,
    framesep=2mm]{c++}
    std::vector<int>& operator|(std::vector<int>& v1, const std::function<void(int&)>& f) {
        for (auto& i : v1) {
            f(i);
        }
        return v1;
    }
\end{minted}

\textbf{不使用范围 for,使用 C++20 简写函数模板:}

\begin{minted}[mathescape,	
    linenos,
    numbersep=5pt,
    gobble=2,
    frame=lines,
    framesep=2mm]{c++}
    std::vector<int>& operator|(auto& v1, const auto& f) {
        std::ranges::for_each(v1, f);
        return v1;
    }
\end{minted}

\textbf{各种其他答案的范式无非就是这些改来改去了,没必要再写。}

\subsection{解析}

很明显我们需要重载管道运算符 $|$,根据我们的调用形式 v $|$ f2 $|$ f,
这种\textbf{链式}的调用,以及根据给出运行结果,我们可以知道,重载函数应当返回 v 的引用,并且 v 会被修改。

v $|$ f2 调用 \mintinline{c++}{operator |},operator $|$ 中使用 f2 遍历了 v 中的每一个元素,然后返回 v 的引用,再 $|$ f。

\section{实现自定义字面量 \_f}

\section{实现 print 以及特化 std::formatter}

\section{给定模板类修改,让其对每一个不同类型实例化有不同 ID}

\section{实现 scope\_guard 类型}

\section{解释 std::atomic 初始化}

\section{throw new MyException}

\section{定义 array 推导指引}

\section{名字查找的问题}

\section{遍历任意聚合类数据成员}

\section{emplace\_back() 的问题}

\section{实现make\_vector()}

\section{关于 return std::move(expr)}

\end{document}